% Created 2018-11-27 Tue 15:31
% Intended LaTeX compiler: pdflatex
\documentclass[15pt]{article}
\usepackage[utf8]{inputenc}
\usepackage[T1]{fontenc}
\usepackage{graphicx}
\usepackage{grffile}
\usepackage{longtable}
\usepackage{wrapfig}
\usepackage{rotating}
\usepackage[normalem]{ulem}
\usepackage{amsmath}
\usepackage{textcomp}
\usepackage{amssymb}
\usepackage{capt-of}
\usepackage{hyperref}
\usepackage[scaled]{helvet}
\renewcommand\familydefault{\sfdefault}
\usepackage[T1]{fontenc}
\usepackage{tabularx}
\usepackage[left=2cm, right=2cm,top=2cm]{geometry}
\author{Kevin Poli}
\date{\today}
\title{}
\hypersetup{
 pdfauthor={Kevin Poli},
 pdftitle={},
 pdfkeywords={},
 pdfsubject={},
 pdfcreator={Emacs 26.1 (Org mode 9.1.14)}, 
 pdflang={English}}
\begin{document}

\tableofcontents

\section{Telescope}
\label{sec:org6e4eb81}
\begin{itemize}
\item Connor O’Hara: Image Processing (cohara1@stevens.edu)
\item Kevin Poli: Application/ Artist Tools Developer (kpoli@stevens.edu)
\item Philip Vitale: Application \& Systems Developer (pvitale@stevens.edu)
\item Brendan von Hofe: Machine Learning (bvonhofe@stevens.edu)
\end{itemize}
\subsection{Advisors:}
\label{sec:org32de14d}
Hong Man (hman@stevens.edu), Jeff Thompson (JThomps4@stevens.edu)
\section{Intro}
\label{sec:org7d4721a}
Telescope is a machine learning assisted toolkit for digital video compositors
with applications in visual effects, matte painting and diverse use cases
accross the video post production pipeline. Tools from existing compositing
packages will interact with a novel ML core to assist or completely automate the
rotoscoping process. Rotoscoping is the process of masking and segmenting
poritons of an image accross multiple moving frames, any feature length movie
will often consist of hundreds of rotoscoped shots with multiple tracked mattes
per image, and this is the primary job of thousands of roto artists accross the world.

We hope to make this process, fast, intuitive, and accesible to alleviate the
manual and time consuming process that makes up a huge chunk of the man hours
required to produce even low budget features. We believe that machine learning
is in the process of revolutionizing image processing, and that user driven
toolkits rather than black box command line workflows will bring our intelligent
core into the hands of the artists where they can thrive.
\subsection{Demonstration}
\label{sec:org13c8a54}
Rotoscoping is the process of frame by frame selecting and isolating a given feature (usually
an object or person) in a video, such that you can produce a video clip of
exclusively that selection on a transparent background

Lets walk through this step by step:

\begin{itemize}
\item First, our source image at frame 1, of Marceu the Mime
\begin{center}
\includegraphics[width=.9\linewidth]{./roto/Capture.PNG}
\end{center}
\item Lets start by creating a selection just of the Mime's face and hand - these
are the features that are actually being "rotoscoped" out
\begin{center}
\includegraphics[width=.9\linewidth]{./roto/masked.PNG}
\end{center}
\item This purple selection represents a 'mask' which are the points and curves that
make up the boundary of what we are looking to isolate. Traditionally, artists
will digitally paint this selection in a software of their choice, by hand.
\item This selection or mask is different from a matte, which is another important
piece of terminology. A matte is a single channel image; meaning rather than
pixels having red,green,blue values, they only contain 1 value from 0-255
called 'alpha'. 'Alpha' will often be displayed in software as white. The
Matte of this selection is an image where only the pixels corresponding to the
selection are white, and all other pixels are black.
\begin{center}
\includegraphics[width=.9\linewidth]{./roto/matte.PNG}
\end{center}
\begin{itemize}
\item this is so that, under the hood, all we need to do is pixel-wise 'multiply' the
source image to the matte, meaning any pixels with a black 'zero value' in
the matte will become transparent, and any pixels in the white '255 value'
in the matte will remain.
\end{itemize}
\begin{center}
\includegraphics[width=.9\linewidth]{./roto/goals.PNG}
\end{center}
\begin{itemize}
\item Here is the result of that multiply, an image containing only the pixels we
selected before
\end{itemize}
\end{itemize}
\subsubsection{Frame By Frame}
\label{sec:orga0d5090}
Much of the challenge and tedium of rotoscoping comes from repeating the above
process for every frame, traditionally, artists will go frame by frame through
the video and manually adjust their selections to match the feature they are
isolating, here is the next frame of that video, with an adjusted selection for
clarity

\begin{center}
\includegraphics[width=.9\linewidth]{./roto/nextframe.PNG}
\end{center}

to see how the selections should move as features in the video move, check out
this gif that displays the matte on the left, with the source on the right, and
has selection lines on both
\url{https://www.google.com/url?sa=i\&rct=j\&q=\&esrc=s\&source=images\&cd=\&cad=rja\&uact=8\&ved=2ahUKEwj4poGei\_bdAhVvTt8KHYSXBs0QjRx6BAgBEAU\&url=https\%3A\%2F\%2Ftaukeke.com\%2F2014\%2F07\%2Frotoscoping-in-nuke\%2F\&psig=AOvVaw0rzB0nhBNxm\_0WD1VdybtL\&ust=1539062086451365}
\subsubsection{Use Cases}
\label{sec:org5fbe941}
With our selection isolated, we can start to play with the image accordingly

By layering the source footage and our rotoscoped hand and face, we can apply an
effect, like the 'colorama' effect to only the pixels we roto'd previously

\begin{center}
\includegraphics[width=.9\linewidth]{./roto/isolated.PNG}
\end{center}
\begin{enumerate}
\item Compositing
\label{sec:orgb1ac4f8}
The most popular use case for rotoscoping is Compositing, which is the process
of combining multiple images into one. Consider three layers to see how this is
done.

Say we want this red square video clip to appear 'behind' the Mime's face and
hand (note what appears black is acutally transparent)

\begin{center}
\includegraphics[width=.9\linewidth]{./roto/red.PNG}
\end{center}

We can grab our source clip and place the square image on top
  \begin{center}
\includegraphics[width=.9\linewidth]{./roto/halfcomp.PNG}
\end{center}
Then grab our rotoscoped face and hand and place that on top
  \begin{center}
\includegraphics[width=.9\linewidth]{./roto/void.PNG}
\end{center}
and here is the desired effect

\begin{center}
\includegraphics[width=.9\linewidth]{./roto/behind.PNG}
\end{center}
\end{enumerate}
\section{Technical Plan}
\label{sec:org333c036}
\subsection{Components}
\label{sec:org2e9d543}
Telescope as a product will consist of two primary modules, the Telescope Core,
which is a machine learning core assisted by traditional algorithmics that
implements the novel functionality of Telescope, and an exchange plugin that
allows existing professional compositing tools to interact with our proccesses.
Telescope For Nuke is our chosen example exhange plugin, designed to demonstrate
how the Telescope core can interact with existing artist workflows - but the
separation of core and plugin is designed such that Telescope can be implemented
into other software packages like Adobe After Effects or Blackmagic Design
Fusion at a later date.
\begin{center}
\begin{tabular}{ll}
Category & What are we using?\\
\hline
Communication & \\
Email & Gmail\\
Web Conferencing & Facebook Video\\
Instant Messaging & GroupMe\\
Collaboration & \\
Document Collaboration & Google Drive\\
File Sharing/Data Tracking & GitHub\\
Plugin Development & \\
OS Supported & Windows, Mac OS, Linux\\
Host Application & Nuke\\
Development Language & C++\\
Machine Learning Development & \\
Development Language & Python\\
Packages & PyTorch\\
\end{tabular}
\end{center}
\subsection{Algorithmics}
\label{sec:orgb1024e3}

The algorithmic core of our plugin will take images (frames of videos) as input and output segmentation masks (mattes) as output. The goal of the masks is to identify all the discrete objects in the image. It is class-agnostic and therefore does not need to determine what the objects are (e.g. cat or dog) but rather the fact that they are discrete.
Our criteria for determining how well our model is accomplishing the task is the Intersection-over-Union metric (IoU). We have yet to determine what an acceptable IoU score is for industry applications.
The model will be a convolutional neural network. Specifically, we will begin with the UNet model (\url{https://arxiv.org/abs/1505.04597}). Initially, our primary dataset to train the model with will be the Panoptic Detection COCO dataset, modified for a class-agnostic task.
Further iterations of the model will take advantage of the additional information in EXR images to refine object mattes and the DAVIS video object segmentation dataset.

\subsection{Dependency Model}
\label{sec:org64e5d9f}
\begin{center}
\includegraphics[width=18cm]{./Dgraph.pdf}
\end{center}
\subsection{Plugin UI Mockups}
\label{sec:org748a618}
\begin{center}
\includegraphics[width=.9\linewidth]{./mockup.png}
\end{center}
\subsection{Deep Learning Core}
\label{sec:org402ee43}

\begin{center}
\includegraphics[width=.9\linewidth]{/Users/kevinpoli/org/.attach/image1_20181114_122910.png}
\end{center}
The core of our rotoscoping program is the deep learning model that takes the image to be cropped and an associated trimap as input (the two leftmost frames respectively), and outputs the cropped portion (rightmost frame). The image to be cropped can be of anything the user wishes. The associated trimap is used to identify the subject (foreground) in the image that the users wishes to crop out. It is drawn with auxiliary tools of our software to identify the definite foreground in white, the unsure foreground (e.g. hair) in grey, and definite background in black. The output is an alpha matte that can be used to crop the subject out of the original image.


\begin{center}
\includegraphics[width=.9\linewidth]{/Users/kevinpoli/org/.attach/image2_20181114_123001.png}
\end{center}
 The deep learning core is defined by the architecture (type of neural network) and its training process. The architecture is composed of two convolutional neural networks. The input data first passes through an encoder-decoder style network, commonly used for segmentation tasks (in the image above, the encoder is the first half of the ‘U’ and the decoder is the second half). Through the successive steps of the network, the image is transformed into different representations called feature maps. The initial representation is the input image itself along with the trimap. As a multidimensional array it has a shape of [height, width, channels]. In this case, there are 4 channels. Three are the RGB channels of the input image, and the last is the associated trimap. As the image passes through the encoder, the feature map representing the image becomes shorter and thinner, but much deeper (e.g. shape of [7, 7, 2048]. The receptive field[1] of the convolutions grows, theoretically allowing it to make higher level abstractions about the subject matter of the image. The decoder then uses this information to eventually generate a single channel image with the same height and width of the original as it uses transposed convolutions.
        The second convolutional neural network is much simpler, composed of only a couple convolutional layers similar to the first of the encoder’s. The input to this network is the original image along with the alpha matte produced from the first network. It outputs a refined version of the alpha matte.
        The training process involves feeding the networks images that we also have ground truth alpha mattes for. After the network outputs a prediction for the alpha matte, we compare it with the ground truth alpha matte using a loss function. This loss function is a combination of two loss functions, the alpha prediction loss, and the compositional loss. The former simply calculates the pixel-wise squared error between mattes. The latter does the same calculation but with the original RGB image composited with the ground truth and predicted alpha mattes. Because of the differentiable nature of the loss functions and backpropagation[2], the weights of the neural network are able to be updated to perform the associated task of rotoscoping slightly better after training on each image. We repeatedly train on the entire dataset until improvements become negligible.
\subsubsection{More Information}
\label{sec:org2c47e66}
\begin{itemize}
\item \url{https://medium.com/mlreview/a-guide-to-receptive-field-arithmetic-for-convolutional-neural-networks-e0f514068807}
\item \url{https://medium.com/datathings/neural-networks-and-backpropagation-explained-in-a-simple-way-f540a3611f5e}
\end{itemize}
\subsection{Completion Schedule}
\label{sec:orgaaa3358}
For this semester, we hope to create a working plugin and corresponding ML core
that allows rotoscoping to be procedureally assisted on a single frame basis.
This does not mean that it will be impossible to rotoscope a video, just that
any given frame will not influence the matte of any other frame e.g matte
interpolation. Matte Interpolation via extending our ML core is our primary goal
for next semester along with testing and final implementations.
\begin{center}
\begin{tabularx}{\textwidth}{XXl}
Applications & Machine Learning & Week\\
\hline
\sout{TriMap Drawing Node} & \sout{Recreating Deep Image Matting Paper} & 10/14\\
\hline
\sout{Investigate Frame By Frame Refinement of TriMap Drawings} & \sout{Continued Implementation, Begin Testing} & 10/21\\
\hline
\sout{Complete UI Skeleton for all Basic Tasks (TriMap Drawing, Selection, Interpolation)} & \sout{Rigorously evaluate limits of model in realistic setting and research improvements} & 10/28\\
\hline
\sout{Containerize Dependencies for Nuke ML Module Loader} & \sout{Dataset creation and augmentation improvements} & 11/4\\
\hline
\sout{Begin Developing ML Module Loader} & \sout{Experiment with different model architectures} & 11/11\\
\hline
\sout{Integrate ML Module Loader into existing plugin Node} & Explore new training schedules and perform hyperparameter tuning & 11/18\\
\hline
Test ML Module Loader on Various Models & Begin research on frame-to-frame interpolation & 11/25\\
\hline
Research TriMap Interpolation & Start implementation & 12/2\\
\hline
Implement Existing Matte to Curves output & Finish implementation & 12/9\\
\hline
Finish up integrations for single frame implementation & Evaluate and improve & 12/6\\
\end{tabularx}
\end{center}


\section{Team}
\label{sec:org3f3ce63}
\subsection{Roles}
\label{sec:org47ffda7}
\begin{itemize}
\item Connor O’Hara: Image Processing (cohara1@stevens.edu)
\item Kevin Poli: Application/ Artist Tools Developer (kpoli@stevens.edu)
\item Philip Vitale: Application \& Systems Developer (pvitale@stevens.edu)
\item Brendan von Hofe: Machine Learning (bvonhofe@stevens.edu)
\end{itemize}



\subsection{Delegation of Tasks}
\label{sec:orgb479d3c}

\subsubsection{Connor O’Hara}
\label{sec:orgcdacdb3}
\begin{enumerate}
\item Last Week
\label{sec:orgaabd651}
\begin{itemize}
\item New, representative and noncyclical dependency graph
\end{itemize}
\item Update
\label{sec:org6e4ffc4}
\begin{itemize}
\item Replaced in its original section
\end{itemize}
\item For Next week
\label{sec:org3b35c22}
\begin{itemize}
\item Research into adapting Matte to Curves for comples mattes like hair or semitransparency
\end{itemize}
\end{enumerate}

\subsubsection{Kevin Poli}
\label{sec:orgb78b303}
\begin{enumerate}
\item Last Week
\label{sec:org452a592}
\begin{itemize}
\item Resuming work in Nuke, and exploring API for novel interactions
\end{itemize}
\item Update
\label{sec:orge3e3cc8}
\begin{itemize}
\item Looking to complete TriMap drawing node with support for edge dialations, masks etc.
\end{itemize}
\item For Next Week
\label{sec:org2339e56}
\begin{itemize}
\item Functional 1-channel 3 color drawing node for TriMaps
\end{itemize}
\end{enumerate}
\subsubsection{Phil Vitale}
\label{sec:org22a2b59}
\begin{enumerate}
\item Last Week
\label{sec:orgf06e107}
\begin{itemize}
\item work will be split into TriMap drawing and module loader nodes
\end{itemize}
\item Update
\label{sec:orgae70728}
\begin{itemize}
\item work has began on the ML module loader
\end{itemize}
\item For Next Week
\label{sec:org6c4ddc0}
\begin{itemize}
\item rough draft of the module loader, may be as simple as shellscript/watch folders
\end{itemize}
\end{enumerate}
\subsubsection{Brendan Von Hofe}
\label{sec:orgc9a3b65}
\begin{enumerate}
\item Last Week
\label{sec:orgc212c7f}
Utilizing dilation and erosion algorithms to automatically generate trimaps from
image mattes. See attached image of dialated trimap

\begin{center}
\includegraphics[width=.9\linewidth]{/Users/kevinpoli/org/.attach/1133x794_20181105_130958.png}
\end{center}
\item Update
\label{sec:orgf5b0dd7}
\begin{itemize}
\item We have begun training the model, initial results even at low training times
are looking good. A high level explaination of our techniques has been
included above.
\end{itemize}
\item For Next Week
\label{sec:orgf512713}
\begin{itemize}
\item continue iterating on the model, and explore options for further fine tuning
\end{itemize}
\end{enumerate}
\end{document}
